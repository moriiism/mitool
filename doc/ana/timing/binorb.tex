%\documentstyle[psfig,11pt]{article} 
\documentstyle[11pt]{jarticle} 
%\documentclass[letter,12pt]{article}
%\documentclass[letter,11pt]{jarticle}

%\usepackage[dvips]{graphicx}

%\nonstopmode

\textwidth=17.8cm
%\textheight=26cm
%\topmargin=-2.5cm
\oddsidemargin=-1.1cm
\evensidemargin=-1.1cm
%\parindent 10pt
%
\def\Ltsim{\mathrel{\vbox to 0pt{\hbox{$\sim$}}\llap{$<$}}}
\def\Gtsim{\mathrel{\vbox to 0pt{\hbox{$\sim$}}\llap{$>$}}}

\title{Orbital motion}
\author{Mikio Morii}
 
\begin{document}
\maketitle
%\medskip
\begin{abstract}
I summarize about Kepler motion.
\end{abstract}

\section{Definition}

現代の天文学シリーズ「天体の位置と運動」に従って、軌道運動について考える。

可視光を放射する星(1)とX線を放射するCompact 天体(2)との連星系を考える。
それぞれの座標を${\bf r}_1$、${\bf r}_2$ とおく。
また、それぞれの質量を$m_1$、$m_2$ とおく。
このとき、星1から測ったCompact天体の相対位置は、
\begin{equation}
{\bf r} = {\bf r}_2 - {\bf r}_1
\end{equation}
である。
重心の座標は、
\begin{equation}
{\bf r}_{\rm G} = \frac{m_1 {\bf r}_1 + m_2 {\bf r}_2}{m_1 + m_2}
\end{equation}
である。
逆に、相対位置${\bf r}$と重心位置${\bf r}_{\rm G}$を用いて、
${\bf r}_1$、${\bf r}_2$ を表すと
\begin{equation}
  {\bf r}_1 = {\bf r}_{\rm G} - \frac{m_2}{m_1 + m_2} {\bf r},
\end{equation}
\begin{equation}
  {\bf r}_2 = {\bf r}_{\rm G} + \frac{m_1}{m_1 + m_2} {\bf r},
\end{equation}
となる。

運動方程式は、
\begin{equation}
m_1 \frac{d^2 {\bf r}_1}{d t^2} = - \frac{G m_1 m_2}{r^3}
({\bf r}_1 - {\bf r}_2),
\end{equation}
\begin{equation}
m_2 \frac{d^2 {\bf r}_2}{d t^2} = - \frac{G m_1 m_2}{r^3}
({\bf r}_2 - {\bf r}_1)
\end{equation}
となる。
これらを引き算すると、
\begin{equation}
\frac{d^2 {\bf r}}{d t^2} = - \frac{\mu}{r^3} {\bf r}
\end{equation}
となる。但し、$\mu = G (m_1 + m_2)$。

Doppler Shift の観測では、重心から測った位置が重要になるので、
\begin{equation}
  {\bf r}_{\rm x} = {\bf r}_2 - {\bf r}_G
  = \frac{m_1}{m_1 + m_2} {\bf r}
\end{equation}
とおく。X線の観測で推定するのは、この${\bf r}_{\rm x}$ である。

\section{..}

これからしばらく、相対運動のベクトル${\bf r}$について考える。
我々観測者から天体を見たときの視線方向を$z_s$軸とする。我々から遠ざかる方向を
プラスとする。天球に接する接平面を$x_s-y_s$平面とする。
また、連星の軌道面を$x-y$平面とする。
連星の軌道面に垂直なベクトル$z$軸と、$z_s$軸との成す角を
$i$ (inclination angle)とおく。

軌道が$x_s-y_s$平面を下($z_s<0$)から上($z_s>0$)に横切る点を昇交点($N$)と呼ぶ。
昇交点の$x_s$軸からの角を昇交点経度$\Omega$とおく。
この値を観測から求めることはできない。

軌道面($x-y$平面)上で、近点(pericenter)の方向を$x$軸とする。
近点の昇交点からの角度を$\omega$とおき、
近点引数(argument of pericenter)と呼ぶ。

離心近点角(eccentric anomaly)$u$, 軌道長半径$a$,
離心率(eccentricity)$e$ を用いて天体$2$の位置を表すと、
p148 (4.48) より、
\begin{equation}
{\bf r} = 
\left( \begin{array}{c}
x \\
y \\
\end{array}
\right)
= a 
\left( \begin{array}{c}
\cos u - e \\
\sqrt{1 - e^2} \sin u \\
\end{array}
\right)
\end{equation}
と表せる。これを時刻で微分し(4.56) を用いると速度が得られる。
\begin{equation}
{\bf v} = 
\left( \begin{array}{c}
v_x \\
v_y \\
\end{array}
\right)
= \frac{n a}{1 - e \cos u} 
\left( \begin{array}{c}
-\sin u \\
\sqrt{1 - e^2} \cos u \\
\end{array}
\right).
\end{equation}

また、
\begin{equation}
  a_{\rm x} = \frac{m_1}{m_1 + m_2} a
\end{equation}
とおくと、
\begin{equation}
{\bf r}_{\rm x} = 
\left( \begin{array}{c}
x_{\rm x} \\
y_{\rm x} \\
\end{array}
\right)
= a_{\rm x}
\left( \begin{array}{c}
\cos u - e \\
\sqrt{1 - e^2} \sin u \\
\end{array}
\right),
\end{equation}
\begin{equation}
{\bf v}_{\rm x} = 
\left( \begin{array}{c}
{v_x}_{\rm x} \\
{v_y}_{\rm x} \\
\end{array}
\right)
= \frac{n a_{\rm x}}{1 - e \cos u} 
\left( \begin{array}{c}
-\sin u \\
\sqrt{1 - e^2} \cos u \\
\end{array}
\right).
\end{equation}


p150より、
近日点通過時刻(time of pericenter passage)を$t_0$ とおくと、
\begin{equation}
  u - e \sin u = n (t - t_0) = l
\end{equation}
と書ける。ここで、$n$は、平均運動(mean motion)で、
\begin{equation}
  n = \sqrt{ \frac{\mu}{a^3} }
\end{equation} 
であり、2体問題では一定の値になる。
$l$は、平均近点角(mean anomaly) と呼ばれる。
mean anomaly は、記号$M$ で表現されることもある。
公転周期$P$は、
\begin{equation}
  P = \frac{2 \pi}{n} = 2 \pi \sqrt{\frac{a^3}{\mu}}
\end{equation}
となる。

p146の図4.4を見て、$x_s-y_s-z_s$空間から$x-y-z$空間への変換を考える。
これは、座標系の変換を3回繰り返すことで得られる。
(B.55) - (B.57) は、 座標系の回転行列なので、これを用いると、
\begin{equation}
\left( \begin{array}{c}
x \\
y \\
z \\
\end{array}
\right)
= R_3(\omega) R_1(i) R_3(\Omega)
\left( \begin{array}{c}
x_s \\
y_s \\
z_s \\
\end{array}
\right)
\end{equation}
と書ける。
今、求めたいのは、$x_s-y_s-z_s$空間での値なので、これを変形すると、
\begin{equation}
\left( \begin{array}{c}
x_s \\
y_s \\
z_s \\
\end{array}
\right)
= R_3(-\Omega) R_1(-i) R_3(-\omega) 
\left( \begin{array}{c}
x \\
y \\
z \\
\end{array}
\right)
\end{equation}
となる。

$\Omega$は観測不能なので、ゼロとおけばよい。
$R_1(-i) R_3(-\omega)$ を計算して、
$({x_s}_{\rm x}, {y_s}_{\rm x}, {z_s}_{\rm x})^{\rm T}$、
$({v_x}_{s \rm x}, {v_y}_{s \rm x}, {v_z}_{s \rm x})^{\rm T}$
を求めることにする。

\begin{equation}
R_1(-i) R_3(-\omega) =
\left( \begin{array}{ccc}
\cos(- \omega) &  \sin(- \omega) & 0 \\
-\cos(-i) \sin(- \omega) &  \cos(-i) \cos(-\omega) & \sin(-i) \\
\sin(-i) \sin(-\omega) & - \sin(-i) \cos(-\omega) & \cos(-i) \\
\end{array}
\right)
\end{equation}
より、
%
\begin{equation}
{x_s}_{\rm x} = a_{\rm x} 
\left[ \cos \omega (\cos u - e)
 - \sin \omega \sqrt{1 - e^2} \sin u
\right],
\end{equation}
%
\begin{equation}
{y_s}_{\rm x} = a_{\rm x} \cos i
\left[ \sin \omega (\cos u - e)
 + \cos \omega \sqrt{1 - e^2} \sin u
\right],
\end{equation}
%
\begin{equation}
{z_s}_{\rm x} = a_{\rm x} \sin i 
\left[ \sin \omega (\cos u - e)
 + \cos \omega \sqrt{1 - e^2} \sin u
\right].
\end{equation}
%
また、
\begin{equation}
v_{z_{s \rm x}} = \frac{a_{\rm x} n \sin i}{1 - e \cos u}
\left[ - \sin \omega \sin u
 + \cos \omega \sqrt{1 - e^2} \cos u
\right].
\end{equation}

\section{Eclipse}

Mean Longitude ($L$)の定義は、
\begin{equation}
L = \Omega + \omega + l.
\end{equation}
ここで、$\Omega$ と($\omega$, $l$)は、異なる平面内での角度である。
それらを大きさだけで足しているので、変な定義である。
Doppler Shift の観測では、$\Omega$は観測できないので、ゼロと置いてよい。

$T_{90}$、$T_{\omega}$という時刻は、それぞれMean Longitude 
が90度になる時刻、$\omega$になる時刻である。
$T_{90}$の時には、compact 星(2)はおおよそ、可視光を放射する星(1)の
背後に位置することになるので、inclination が大きければX線のEclipse が
起こりそうでである。しかし、Eclipse はこの時刻ではない。
$T_{\omega}$ は、近点通過の時刻なので、$t_0$に等しい。

Eclipse は、
\begin{equation}
  \omega + f = 90 {\rm deg}
\end{equation}
になるときである。ここで$f$は、真近点角(true anomaly)である。


さて、
\begin{equation}
L = n(t - t_0) + \omega
\end{equation}
より、
\begin{equation}
t = \frac{L - \omega}{n} + t_0
\end{equation}
なので、
\begin{equation}
T_{90} - T_{0} = \frac{90^\circ - \omega}{n} + t_0 -
(- \frac{\omega}{n} + t_0) = \frac{90^\circ}{n} = \frac{P}{4}
\end{equation}
の関係が成り立つ。


\section{時刻補正}

compact星がパルサーの場合、パルスを検出したいとする。
太陽系のbarycenter で 観測されたphotonの到来時刻
($t_{\rm obs}^{\rm (solar-bary)}$)を
パルサーから放出された時刻($t_{\rm obj}$)に変換しないといけない。
まず、太陽系のbarycenter と binary system の重心との距離は観測中は
不変と考える。その距離をphotonは移動してきたわけだが、
その値は観測中一定なので、無視してゼロと置く。
つまり、
$t_{\rm obs}^{\rm (solar-bary)} = t_{\rm obs}^{\rm (binary-bary)}$。
以降、これらを簡単に$t_{\rm obs}$と表記することにする。

binary system の重心からパルサーまでの視線方向の奥行きは、
${z_s}_{\rm x}$である。$t_{\rm obj}$の時刻に放出されたphotonは、
${z_s}_{\rm x}/c$だけの時間をかけて、$t_{\rm obs}$ の時刻に、
binary system の重心に到達する。
つまり、$t_{\rm obs} = t_{\rm obj} + {z_s}_{\rm x}/c$である。
ただし、${z_s}_{\rm x}$は、時刻$t_{\rm obj}$の関数である。
従って、
\begin{equation}
t_{\rm obs} = t_{\rm obj} + \frac{{z_s}_{\rm x}(t_{\rm obj})}{c}.
\end{equation}
ここで、
$t_{\rm obs} \neq t_{\rm obj} + \frac{{z_s}_{\rm x}(t_{\rm obs})}{c}$
であることに注意しないといけない。
$t_{\rm obj}$を得るには、
$f(t_{\rm obj}; t_{\rm obs}) =
t_{\rm obj} + \frac{{z_s}_{\rm x}(t_{\rm obj})}{c} - t_{\rm obs}$
と置いて、方程式$f(t_{\rm obj}; t_{\rm obs}) = 0$の根を求める必要がある。

\end{document}

